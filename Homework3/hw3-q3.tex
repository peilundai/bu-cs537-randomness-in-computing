\documentclass[10pt]{537homework}

% For including image files
\usepackage{graphicx}
\usepackage[ruled,vlined,noline]{algorithm2e}
% set the vertical spacing between paragraphs
\setlength{\parskip}{1.5mm}

% For fancy math
\RequirePackage{amsmath,amsthm,amssymb}
\newtheorem{theorem}{Theorem}
\newtheorem{fact}[theorem]{Fact}
\newtheorem{lemma}[theorem]{Lemma}
\newtheorem{claim}[theorem]{Claim}
%\usepackage{amsmath}
\usepackage{titlesec}
\titleformat{\subsection}[runin]{}{}{}{}[]

\newcommand{\ord}[2][th]{\ensuremath{{#2}^{\mathrm{#1}}}}
% shorthand for \mathcal{O}
\newcommand{\Ocal}{\ensuremath{\mathcal{O}}}


% homework number
\hwnumber{3}
% problem number
\problemnumber{3}
% your name
\author{Peilun Dai}
% Collaborators. If you didn't collaborate, write "\collaborators{none}".
% If you did, for each collaborator, write "worked together", "I helped him/her" or "He/she helped me".
 \collaborators{None}


\begin{document}
\section*{3. (Jensen's Inequality) }

%%%%%%%%%%%%%%%%%%%%%%%%%%%%%%%%%%%%%%%%%%%%%%%%%%

\subsection*{(a)} Solutions: If $f$ is a concave function, then 

\begin{align}
  \mathbb{E}(f(X)) & \leq f(\mathbb{E}(X))
\end{align}

This follows naturally from the original Jensen's inequality because if $f$ is concave, then $-f$ will be convex, and we can apply the Jensen's equality to $-f$, we would obtain the equality for concave function $f$ as above. 

%%%%%%%%%%%%%%%%%%%%%%%%%%%%%%%%%%%%%%%%%%%%%%%%%%

\subsection*{(b)} \proof 

Let $G_n = \sqrt[n]{\prod_{i=1}^n x_i}$ be the geometric mean and $A_n = {1\over n}\sum_{i=1}^n x_i$ be the arithmetic mean of a collection of $n$ positive real numbers $\{x_i\}$.

\begin{align}
  \log{A_n}   & = \log{\Big({1 \over n} \sum_{i=1}^n x_i \Big)} \\
              & \geq {1 \over n} \sum_{i=1}^n \log{x_i} \\
              & = \sum_{i=1}^n \Big( \log{x_i^{1/n}} \Big) \\
              & = \log{\Big( \prod_{i=1}^n x_i^{1/n} \Big)} \\
              & = \log{G_n} \\
        A_n   & \geq G_n 
\end{align}

From (2) to (3) we have used the inequality in (a) where $f(x) = \log{x}$. (7) is obtained when we take the exponential of $\log{A_n}$ and $\log{G_n}$ respectively. 

%%%%%%%%%%%%%%%%%%%%%%%%%%%%%%%%%%%%%%%%%%%%%%%%%%

\subsection*{(c)} \proof 

let $f(x) = \sin{x}$ where $ 0 < x < \pi$. Because $f``(x) = - \sin{x} < 0$ when $0 < x < \pi $, $f(x)$ is concave in the interval $(0, \pi)$. We can apply the inequality in (a),

\begin{align}
  {1 \over 3}\Big(\sin{A} + \sin{B} + \sin{C} \Big) & \leq \sin{{1\over3}(A + B + C)} \\
  & \leq \sin{60^{\circ}} \\
  & \leq {\sqrt{3} \over 2} \\
  \sin{A} + \sin{B} + \sin{C} & \leq {3\sqrt{3} \over 2} 
\end{align} 


\end{document} 

















































