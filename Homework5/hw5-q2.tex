\documentclass[11pt]{537homework}

% For including image files
\usepackage{graphicx}
\usepackage[ruled,vlined,noline]{algorithm2e}
% set the vertical spacing between paragraphs
\setlength{\parskip}{1.5mm}

% For fancy math
\RequirePackage{amsmath,amsthm,amssymb}
\newtheorem{theorem}{Theorem}
\newtheorem{fact}[theorem]{Fact}
\newtheorem{lemma}[theorem]{Lemma}
\newtheorem{claim}[theorem]{Claim}
%\usepackage{amsmath}
\usepackage{titlesec}
\titleformat{\subsection}[runin]{}{}{}{}[]

\newcommand{\ord}[2][th]{\ensuremath{{#2}^{\mathrm{#1}}}}
% shorthand for \mathcal{O}
\newcommand{\Ocal}{\ensuremath{\mathcal{O}}}


% homework number
\hwnumber{5}
% problem number
\problemnumber{2}
% your name
\author{Peilun Dai}
% Collaborators. If you didn't collaborate, write "\collaborators{none}".
% If you did, for each collaborator, write "worked together", "I helped him/her" or "He/she helped me".
 \collaborators{None}


\begin{document}
\section*{2. (MaxCut) }


%%%%%%%%%%%%%%%%%%%%%%%%%%%%%%%%%%%

\subsection*{(a)}

Let $X_{ij}$ denote the Bernoulli random variable whether the edge $\{i,j\}$ is in the cut set $S$, where $\{i,j\} \in E$, $i<j$ and $m=|E|$ is the total number of edges in the graph. Then $X = \sum_{\{i, j\} \in E} X_{ij}$. For R.V. $\{i,j\}$, since it is a Bernoulli random variable, $\mathbf{E}(X_{ij}) = Pr[X_{ij}=1] = 1/2$. Thus,
\begin{align}
  \mathbf{E}[X]   & = \mathbf{E}\Big[\sum_{\{i, j\} \in E} X_{ij}\Big] \\
                  & = \sum_{\{i, j\} \in E} \mathbf{E}[X_{ij}] \\
                  & = \sum_{\{i, j\} \in E} Pr[X_{ij} = 1] \\
                  & = \sum_{\{i, j\} \in E} {1 \over 2} \\
                  & = {1 \over 2}m
\end{align}

It is easy to show that $OPT \leq m$ since the size of max cut cannot exceed the total number of edges in the graph, thus $\mathbf{E}(X) = {1 \over 2}m \geq {OPT \over 2}$.

%%%%%%%%%%%%%%%%%%%%%%%%%%%%%%%%%%%

\subsection*{(b)}

\proof Let R.V. $Y$ denote the number of edges not in the cut set $S$, then $Y = m - X$. 
\begin{align}
  Pr[X \geq 0.49 OPT]   & = Pr[Y \leq m - 0.49 OPT] \\
                        & = 1 - Pr[Y \geq m - 0.49 OPT] \\
                        & \geq 1 - {\mathbf{E}(Y) \over m - 0.49 OPT} \\
                        & = 1 - {m \over 2(m - 0.49 OPT)} \\
                        & \geq 1 - {m \over 2(0.51m) } \\
                        & = 1 - {1 \over 2 * 0.51}  = {1 \over 51}
\end{align}

From (7) to (8), we have used Markov Inequality. From (8) to (9), we have used the fact that $\mathbf{E}(Y) = m - \mathbf{E}(m) = 0.5m$. And from (9) to (10), we used the obvious fact that $OPT \leq m$. 

%%%%%%%%%%%%%%%%%%%%%%%%%%%%%%%%%%%

\subsection*{(c)}

Similar to (a), we write $X = \sum_{\{i, j\} \in E}X_{ij}$. 
\begin{align}
  Var[X]      & = \mathbf{E}[X^2] - \mathbf{E}(X)^2 \\
              & = \mathbf{E}\Big[\big(\sum_{\{i, j\} \in E}X_{ij} \big)^2\Big] - (0.5m)^2 \\
              & =  \mathbf{E}\Big[\sum_{\{i, j\} \in E}X_{ij}^2 + \sum_{\{i,j\} \in E, \{i, j\} \neq \{m, n\}}\sum_{\{m,n\} \in E}X_{ij}X_{mn}\Big] - (0.5m)^2 \\
              & = \sum_{\{i, j\} \in E} \mathbf{E}\big(X_{ij}^2\big) + \sum_{\{i,j\} \in E, \{i, j\} \neq \{m, n\}}\sum_{\{m,n\} \in E} \mathbf{E}[X_{ij}X_{mn}] - 0.25m^2 \\
              & = m/2 + {m \choose 2}{1\over 4} - 0.25m^2 = m/4
\end{align}

From (14) to (15), we used linearity of expectation, and from (15) to (16), we used the fact that $X_{ij}^2$ follows the same distribution as $X_{ij}$, thus $\mathbf{E}(X_{ij}^2) = \mathbf{E}(X_{ij})$ and that $\mathbf{E}(X_{ij}X_{mn}) = 1/4$ if the two edges are different edges and there are ${m \choose 2}$ such pairs of edges. 

%%%%%%%%%%%%%%%%%%%%%%%%%%%%%%%%%%%

\subsection*{(d)}
\proof 
\begin{align}
  p                     & = Pr[X \leq 0.49 OPT] \\
                        & = 1 - Pr[Y \geq m - 0.49 OPT] \\
                        & = 1 - Pr[Y - 0.5m \geq 0.5m - 0.49 OPT] \\
                        & \geq 1 - Pr[|Y - \mathbf{E}(Y)| \geq 0.5m - 0.49 OPT] \\
                        & \geq 1 - {Var[Y] \over (0.5m - 0.49 OPT)^2} \\
                        & \geq 1 - {m \over 4 (0.01m)^2} \\
                        & = 1 - {2500 \over m} = 1 - 2500 \cdot {1 \over |E|} \\
 p                      & = \Omega(1 - 2500 \cdot {1 \over |E|} ) \\
                        & = 1 - O(1/|E|)
\end{align}
From (19) to (20), we used the fact that $0.5 - 0.49 OPT > 0$. From (20) to (21), we used Chebyshev's inequality. From (21) to (22), we used $Var[Y] = Var[m-X] = Var[X]$ and $OPT \leq m$. 

\end{document} 

















































