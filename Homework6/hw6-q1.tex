\documentclass[11pt]{537homework}

% For including image files
\usepackage{graphicx}
\usepackage[ruled,vlined,noline]{algorithm2e}
% set the vertical spacing between paragraphs
\setlength{\parskip}{1.5mm}

% For fancy math
\RequirePackage{amsmath,amsthm,amssymb}
\newtheorem{theorem}{Theorem}
\newtheorem{fact}[theorem]{Fact}
\newtheorem{lemma}[theorem]{Lemma}
\newtheorem{claim}[theorem]{Claim}
%\usepackage{amsmath}
\usepackage{titlesec}
\titleformat{\subsection}[runin]{}{}{}{}[]

\newcommand{\ord}[2][th]{\ensuremath{{#2}^{\mathrm{#1}}}}
% shorthand for \mathcal{O}
\newcommand{\Ocal}{\ensuremath{\mathcal{O}}}


% homework number
\hwnumber{6}
% problem number
\problemnumber{1}
% your name
\author{Peilun Dai}
% Collaborators. If you didn't collaborate, write "\collaborators{none}".
% If you did, for each collaborator, write "worked together", "I helped him/her" or "He/she helped me".
 \collaborators{None}


\begin{document}
\section*{1. (Improving guarantees of randomized algorithms) }


%%%%%%%%%%%%%%%%%%%%%%%%%%%%%%%%%%%

\subsection*{(a)}

Solution: Let $X$ denote the number of times needed to get the first non-fail outputs for an input of size $n$ and let $T_n = O(n^2)$ be the time that a single run of the algorithm takes to get an output (either``fail" or correct output). Then the total time of getting a correct answer $T(n) = XT_n$. Let $p \geq 1-0.99 = 0.01$ be the probability that the algorithm will output a correct answer in a single run. Then we have $X \sim Geom(p)$. Thus
\begin{align}
  \mathbf{E}[T(n)]  & = \mathbf{E}[X]O(n^2) \\
                    & = {1 \over p} O(n^2) \\
                    & \leq {1 \over 0.01} O(n^2) \\
                    & = O(n^2)
\end{align}
Note that we only stop our runs of the algorithm when we have a correct output, thus this new algorithm always computes a correct answer and runs in expected time $O(n^2)$.

%%%%%%%%%%%%%%%%%%%%%%%%%%%%%%%%%%%

\subsection*{(b)} 

Solution: Let $X$ be the random variable of the running time of algorithm $\mathcal{B}$ on an input of size $n$. The using Markov inequality, for any positive number $b$,
\begin{align}
  Pr[X\geq b] & \leq {\mathbf{E}[X] \over b} \\
              & = {T(n) \over b} \\
              & \leq 1 - 0.95 = 0.05 
\end{align}

We can solve for $b \geq 20T(n)$. Thus we can choose $a = 20$ and run the algorithm $\mathcal{B}$ for $ 20\cdot T(n)$ time, and the failure probability is less than $0.05$ (i.e. with probability at least 0.95, the new algorithm can solve the problem). 


If we know the variable of the algorithm to be at most $\sqrt{n}$, then we can use Chebyshev inequality instead of Markov inequality, for any $b>0$,

\begin{align}
  Pr[ |X - \mathbf{E}[X]| \geq b] & \leq {\mathbf{Var}[X] \over b^2} \\
  Pr[ X - \mathbf{E}[X] \geq b]   & \leq {\mathbf{Var}[X] \over b^2} \\
  Pr[ X \geq T(n) + b]   & \leq {\mathbf{Var}[X] \over b^2} \\
                                  & \leq {\sqrt{n} \over b^2} \leq 0.05\\
\end{align}
We can solve for $b \geq 2\sqrt{5}n^{1/4}$. Thus, we can let the $\mathcal{B}$ algorithm run $T(n) + 2\sqrt{5}n^{1/4}$ time, and the success probability will be at least 0.95. 


%%%%%%%%%%%%%%%%%%%%%%%%%%%%%%%%%%%

\subsection*{(c)} 

Solution: For the algorithm to succeed if we run $\mathcal{C}$ algorithm $k$ times, since we don't know if the incorrect solutions are the same, for the worst case, we need the number of correct solutions to be at least $0.5k$. Then the possible failure case is when the number of correct solution to be less than $0.5k$. Let $X_i$ to be an indicator variable for correct solution returned at run $i$, and let $X = \sum_{i}^k X_i$ be the random variable of the number of correct solutions in $k$ runs. $\mathbf{E}[X] = 0.7k$. Using Chernoff bounds, for $\delta = 2/7$, 
\begin{align}
  Pr[X \leq (1-\delta)\mathbf{E}[X]]  & \leq e^{-\mathbf{E}[X]\delta^2/2} \\
  Pr[X \leq (1-2/7)0.7k]  & \leq e^{-0.7k(2/7)^2/2} \\
  Pr[X \leq 0.5k]         & \leq e^{-k/35} \leq e^{-t} \\
                    -k/35 & \leq -t \\
                        k & \geq 35t
\end{align}
Thus, if we want the algorithm to make a mistake in computing $f(x)$ with probability at most $2^{-t}$, then we need to set $k \geq 35t$. 


\end{document} 

















































