\documentclass[11pt]{537homework}

% For including image files
\usepackage{graphicx}
\usepackage[ruled,vlined,noline]{algorithm2e}
% set the vertical spacing between paragraphs
\setlength{\parskip}{1.5mm}

% For fancy math
\RequirePackage{amsmath,amsthm,amssymb}
\newtheorem{theorem}{Theorem}
\newtheorem{fact}[theorem]{Fact}
\newtheorem{lemma}[theorem]{Lemma}
\newtheorem{claim}[theorem]{Claim}
%\usepackage{amsmath}
\usepackage{titlesec}
\titleformat{\subsection}[runin]{}{}{}{}[]

\newcommand{\ord}[2][th]{\ensuremath{{#2}^{\mathrm{#1}}}}
% shorthand for \mathcal{O}
\newcommand{\Ocal}{\ensuremath{\mathcal{O}}}


% homework number
\hwnumber{6}
% problem number
\problemnumber{2}
% your name
\author{Peilun Dai}
% Collaborators. If you didn't collaborate, write "\collaborators{none}".
% If you did, for each collaborator, write "worked together", "I helped him/her" or "He/she helped me".
 \collaborators{None}


\begin{document}
\section*{2. (Exercise 4.10 from MU) }


%%%%%%%%%%%%%%%%%%%%%%%%%%%%%%%%%%%

\subsection*{(a)} 

Solution: Let R.V. $X_i$ be the amount of money the casino lost in game $i$. $\mathbf{E}(X_i) = (3-1){4 \over 25} + (100-1){1\over 200} + (1)(1 - {4 \over 25} - {1 \over 200}) = -0.02$. In expectation, in each game, the casino will make $0.02$ dollars. Let $X$ denote the total amount of loss the casino has in the first 1 million games, i.e. $X = \sum_{i=1}^{1,000,000} X_i$ with $\mathbf{E}[X] = 1,000,000\cdot(-0.02) = -20,000$. Using Theorem 4.12, $a = -1, b=99$, we can set $\epsilon = 0.03$, 
\begin{align}
  Pr\Big[ {1\over 1,000,000} \sum_{i=1}^{1,000,000} X_i - \mu \geq \epsilon\Big] & \leq e^{-2n\epsilon^2/(b-a)} \\
  Pr\Big[  {1\over 1,000,000} X - (-0.02)   \geq 0.03 \Big] & \leq e^{-0.18} \\
  Pr\Big[ X \geq 10,000 \Big] \leq e^{-0.18} = 0.84\\
\end{align}


%%%%%%%%%%%%%%%%%%%%%%%%%%%%%%%%%%%

\subsection*{(b)} 

Solution: Since $X_i$ are mutually independent,

\begin{align}
  \mathbf{E}[e^{tX}]      & = \mathbf{E}\Big[\exp \big\{t\sum_{i=1}^{1,000,000} X_i\big\}\Big] \\
                          & = \mathbf{E}\Big[\prod_{i=1}^{1,000,000}\exp \big\{tX_i\big\}\Big]\\
                          & = \prod_{i=1}^{1,000,000}\mathbf{E}[e^{tX_i}] \\
                          & = \prod_{i}^{1,000,000}\sum_{X_i \in \{-1, 2, 99\}} Pr[X_i]e^{tX_i} \\
                          & = \prod_{i}^{1,000,000}\Big( {4 \over 25}e^{2t} + {1 \over 200}e^{99t} + {167 \over 200}e^{-t}\Big)
\end{align}


%%%%%%%%%%%%%%%%%%%%%%%%%%%%%%%%%%%

\subsection*{(c)} 

Solution: 

\begin{align}
  Pr[X \geq 10,000]   & = Pr[e^{tX} \geq e^{10,000t}]  \\
                      & \leq {\mathbf{E}(e^{tX}) \over e^{10,000t}} \\
                      & = {\prod_{i}^{1,000,000}\Big( {4 \over 25}e^{2t} + {1 \over 200}e^{99t} + {167 \over 200}e^{-t}\Big) \over e^{10,000t}} \\
                      & = {\prod_{i}^{1,000,000}\Big( {4 \over 25}e^{2\cdot0.0006} + {1 \over 200}e^{99\cdot 0.0006} + {167 \over 200}e^{-0.0006}\Big) \over e^{10,000\cdot 0.0006}} \\
                      & \approx 0.000160646
\end{align}
where we have used $t=0.0006$ and also results from (9). Notice this is a march tighter bound compared to the bound in (a). 


\end{document} 

















































