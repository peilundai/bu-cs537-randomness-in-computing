\documentclass[11pt]{537homework}

% For including image files
\usepackage{graphicx}
\usepackage[ruled,vlined,noline]{algorithm2e}
% set the vertical spacing between paragraphs
\setlength{\parskip}{1.5mm}

% For fancy math
\RequirePackage{amsmath,amsthm,amssymb}
\newtheorem{theorem}{Theorem}
\newtheorem{fact}[theorem]{Fact}
\newtheorem{lemma}[theorem]{Lemma}
\newtheorem{claim}[theorem]{Claim}
%\usepackage{amsmath}
\usepackage{titlesec}
\titleformat{\subsection}[runin]{}{}{}{}[]

\newcommand{\ord}[2][th]{\ensuremath{{#2}^{\mathrm{#1}}}}
% shorthand for \mathcal{O}
\newcommand{\Ocal}{\ensuremath{\mathcal{O}}}


% homework number
\hwnumber{7}
% problem number
\problemnumber{1}
% your name
\author{Peilun Dai}
% Collaborators. If you didn't collaborate, write "\collaborators{none}".
% If you did, for each collaborator, write "worked together", "I helped him/her" or "He/she helped me".
 \collaborators{None}


\begin{document}
\section*{1. (Using Chernoff/Hoeffding bounds) }


%%%%%%%%%%%%%%%%%%%%%%%%%%%%%%%%%%%

\subsection*{(a) (\textbf{Amplification of the success probability})}

Solutions: let $t$ be the number of times we run the algorithm $\mathcal{A}$. Let $X_i$ be the indicator variable that the $i_{th}$ run produces a results that is outside the reasonable range. Let $Y_i$ be the indicator random variable that the $i_{th}$ run produces a result that falls on the left side of the range and let $Z_i$ be an indicator random variable that the $i_{th}$ run produces a result that falls to the right of the reasonable range. Assume we run the algorithm $\mathcal{A}$ $t$ times and let $X = \sum_{i=1}^t$, $Y = \sum_{i=1}^t$ and $Z = \sum_{i=1}^t$, and note that $\mathbf{E}[X] \leq 1/3$, $X = Y + Z$, and $\mathbf{E}[Y] \leq \mathbf{E}[X]$ and $\mathbf{E}[Z] \leq \mathbf{E}[X]$. If our algorithm returns the middle number of these $t$ runs as the results, then our algorithm fails if either $Y > t/2$ or $Z > t/2$. For simplicity, we use $Y \geq t/2$ and $Z \geq t/2$. Using Chernoff bounds, 

\begin{align}
  Pr[Y > t/2]   & \leq Pr[Y \geq t/2] \\
                & = Pr[Y \geq (1+1/2)(t/3)] \\
                & = Pr[Y \geq (1+1/2)\mathbf{E}[X]] \\
                & \leq Pr[Y \geq (1+1/2)\mathbf{E}[Y]] \\
                & \leq e^{-\mathbf{E}[Y](1/2)^2/3} \\
\end{align}  

Note that from (4) to (5), we have used Chernoff bounds. Similarly, for $Z$, we can derive,

\[
Pr[Z > t/2] \leq e^{-\mathbf{E}[Z](1/2)^2/3}
\]

Based on the information 

Thus, Let $F$ denote the event that the modified algorithm fails, using union bound, the fail probability is upper bounded by the sum of the above two probabilities, i.e.

\begin{align}
  Pr[F]   & \leq Pr[Y > t/2] + Pr[Z > t/2] \\
          & \leq e^{-\mathbf{E}[Y](1/2)^2/3} + e^{-\mathbf{E}[Z](1/2)^2/3} \\
          & = e^{-\mathbf{E}[Y](1/2)^2/3} + e^{-(t/3 - \mathbf{E}[Y])(1/2)^2/3} \\
          & = e^{-tp_Y/12} + e^{-t(1/3-p_Y)/12} \\
          & \leq \delta 
\end{align}

where $0\leq p_Y \leq 1/3$ is the probability that a result falls to the left of the range. Without loss of generality, let's assume that $p_Y = 1/2 p_X = (1/2)(1/3) = 1/6$, that is the probability of a sample falling on the left side of the range is same as the probability that it falls on the right side of the range. If we solve the above inequality $(9) \leq (10)$, we can get $t \geq 72\ln{2 \over \delta}$. Thus means if we run the algorithm $\Theta(\log{1 \over \delta})$, i.e. total time $O(T(n)\log{1 \over \delta})$, and take the median of the results, the failure probability is less than $\delta$, i.e. the probability of a good approximation is at least $1-\delta$. 


%%%%%%%%%%%%%%%%%%%%%%%%%%%%%%%%%%%

\subsection*{(b) (\textbf{Coronavirus resilience})}

\proof From the given inequality,$s \geq \max{\Big\{{2m \over \alpha}}, {8 \ln{(1\delta)} \over \alpha} \Big\}$, we also know that $s\alpha \geq 2m$ or $m \leq (\alpha s)/2$, which we will use later. Now let $X_i$ be an indicator random variable that sample $i$ is coronavirus-resilient and let $X = \sum_{i}^s X_i$ be the number of coronavirus-resilient people in the $s$ samples. We want $Pr[X \geq m] \geq 1-\delta$ where $\delta$ is a number between 0 and 1. 

\begin{align}
  Pr[X \geq m]    & = 1 - Pr[X < m] \\
                  & \geq 1 - Pr[X \leq m] \\
                  & = 1 - Pr\Big[X \leq \big(1 - (1 - {m \over \mu})\mu) \big)\Big] \\
                  & \geq 1 - e^{-\mu (1-m/\mu)^2/2}
\end{align}

where $\mu = \mathbf{E}(X) = \alpha s$. One sufficient condition for $Pr[X \geq m] \geq 1 - \delta$ is that $1 - e^{-\mu (1-m/\mu)^2/2} \geq 1 - \delta$ or after simplifying, 

\begin{align}
  \mu\big( 1 - {m \over \mu} \big)^2 \geq 2 \ln{{1 \over \delta}}
\end{align}



Let's first assume that $s \geq {2m \over \alpha} \geq {8 \ln{(1\delta)} \over \alpha}$, or $\mu \geq 2m \geq 8 \ln{(1/\delta)}$. 

\begin{align}
  \mu\big( 1 - {m \over \mu} \big)^2  & \geq 2m(1 - {m \over 2m})^2 \\
                                      & = {m \over 2} \\
                                      & \geq 2\ln{(1/\delta)}
\end{align}

which shows that (16) is satisfied, thus we have a sufficient condition for the original inequality. 

Then let's consider another possibility, $s  \geq {8 \ln{(1\delta)} \over \alpha}\geq {2m \over \alpha}$, or $\mu \geq 8 \ln{(1/\delta)}\geq 2m$. Again (16) is a sufficient condition we want to prove. 

\begin{align}
  \mu(1 - {m \over \mu})^2 & \geq 8\ln{(1/\delta)}(1 - {m \over 8 \ln{(1/\delta)}})^2 \\ 
  & \geq 8\ln{(1/\delta)}(1 - {m \over 2m})^2 \\
  & \geq 8\ln{(1/\delta)}1/4 = 2\ln{(1/\delta)}
\end{align}

which also shows that the sufficient condition for our proof (16) is satisfied. Thus, in either case, as long as we have, $s \geq \max{\Big\{{2m \over \alpha}}, {8 \ln{(1\delta)} \over \alpha} \Big\}$, our original claim is guaranteed. 

\end{document} 

















































