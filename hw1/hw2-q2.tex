\documentclass[10pt]{537homework}

% For including image files
\usepackage{graphicx}
\usepackage[ruled,vlined,noline]{algorithm2e}
% set the vertical spacing between paragraphs
\setlength{\parskip}{1.5mm}

% For fancy math
\RequirePackage{amsmath,amsthm,amssymb}
\newtheorem{theorem}{Theorem}
\newtheorem{fact}[theorem]{Fact}
\newtheorem{lemma}[theorem]{Lemma}
\newtheorem{claim}[theorem]{Claim}
%\usepackage{amsmath}

\newcommand{\ord}[2][th]{\ensuremath{{#2}^{\mathrm{#1}}}}
% shorthand for \mathcal{O}
\newcommand{\Ocal}{\ensuremath{\mathcal{O}}}


% homework number
\hwnumber{1}
% problem number
\problemnumber{2}
% your name
\author{Peilun Dai}
% Collaborators. If you didn't collaborate, write "\collaborators{none}".
% If you did, for each collaborator, write "worked together", "I helped him/her" or "He/she helped me".
% \collaborators{John Doe (worked together), Ben Bitdiddle (I helped him)}

\begin{document}
\section*{2. Homework Assignments, 10 points }


% If the problem has multiple parts, use \subsection command.
\subsection{Solution:}

\begin{proof}
We can prove this by induction. 
When $k=1$, it is obvious that the probability of working on it is 100\%; When $k=2$, the probability of switching to second homework is $50\%$ and of staying on homework 1 is also $50\%$. Thus, the  statement we need to prove is true for $k=1, 2$. Now, for induction, we assume that the statement is true for $k \geq 2$, that is, the probability of working on any of the $k$ homework is $1/k$. Now, when the $(k+1)th$ homework arrives, there is a probability of ${1\over k+1}$ that you will switch to it no matter which homework you are working on right now. Thus, by law of total probability:

\begin{equation*}
	\begin{aligned}
		Pr [(working\:on\:homework\:(k+1))]  = \Big( &\sum_{i=1}^k Pr [(work\:on\:homework\:i)]\cdot \\
		& Pr[(switching\:to\:homework\:(k+1))\:|\: (work\:on\:homework\:i)]\Big) \\
		= & \sum_{i=1}^k {1 \over i}\cdot {1 \over k+1} = {1 \over k+1} \\
	\end{aligned}
\end{equation*}

Now let's also show that the probability of staying in any of the first $k$ homework, $i$, is also ${1\over k+1}$:
\begin{equation*}
	\begin{aligned}
		Pr [(staying\:on\:homework\:(i))]   = &  \Big(1 - Pr[(switching\:to\:homework\:(k+1))\:|\: (work\:on\:homework\:i)]\Big) \cdot \\
		 & Pr [(work\:on\:homework\:i)]  \\
		= &  \Big( 1 - {1\over k+1} \Big) \cdot {1 \over k} \\
		= & {1 \over k + 1}
	\end{aligned}
\end{equation*}

We have shown that when a new homework $k+1$ is assigned, the probability of working on each homework is still equal. Thus, by induction, we have shown that you are equally likely to work on any homework assigned so far. 
	
\end{proof}

\subsection{Solution:}  

Let $k$ be the total number of homeworks assigned so far. and let $Pr_k[i]$  be the probability of working on homework $i$ when total number of homeworks assigned so far is $k$. It is obvious that $Pr_1[1] = 1$, and $Pr_2[1] = Pr_2[2] = 1/2$. 

According to the rules specified, when homework $k$ is assigned, the probability of switching to it is $1/2$ no matter which homework you are working on now. Thus, by law of total probability, for any $k \geq 2$, we have 

$$Pr_k[k]=1/2, Pr_k[1] = Pr_k[2]$$ and 

$$Pr_k[i] = 2\cdot Pr_k[i-1]$$ for $i$ and $k$ satisfying $ 3 \leq i \leq k $. Of course

$$\sum_{i=1}^k Pr_k[i] = 1$$ The following distribution is the only distribution satisfying all the requirements and is thus the distribution we want to obtain

When $k=1$, 

$$Pr_1[1] = 1$$

When $k\geq2$,

\[
Pr_k[i] = 
		\begin{cases}
            {1 \over 2^{k-1}} & \quad i=1, 2 \\[1em]   %%% <--- here
            {1 \over 2^{k-i+1}} & \quad 3 \leq i \leq k
        \end{cases}
\]


\end{document} 