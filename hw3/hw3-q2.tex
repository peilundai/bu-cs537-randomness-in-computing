\documentclass[10pt]{537homework}

% For including image files
\usepackage{graphicx}
\usepackage[ruled,vlined,noline]{algorithm2e}
% set the vertical spacing between paragraphs
\setlength{\parskip}{1.5mm}

% For fancy math
\RequirePackage{amsmath,amsthm,amssymb}
\newtheorem{theorem}{Theorem}
\newtheorem{fact}[theorem]{Fact}
\newtheorem{lemma}[theorem]{Lemma}
\newtheorem{claim}[theorem]{Claim}
%\usepackage{amsmath}
\usepackage{titlesec}
\titleformat{\subsection}[runin]{}{}{}{}[]

\newcommand{\ord}[2][th]{\ensuremath{{#2}^{\mathrm{#1}}}}
% shorthand for \mathcal{O}
\newcommand{\Ocal}{\ensuremath{\mathcal{O}}}


% homework number
\hwnumber{3}
% problem number
\problemnumber{2}
% your name
\author{Peilun Dai}
% Collaborators. If you didn't collaborate, write "\collaborators{none}".
% If you did, for each collaborator, write "worked together", "I helped him/her" or "He/she helped me".
 \collaborators{None}


\begin{document}
\section*{2. (Improved Randomized Min-Cut Algorithm) }

%%%%%%%%%%%%%%%%%%%%%%%%%%%%%%%%%%%%%%%%%%%%%%%%%%%%%

\subsection*{(a)} Solutions: If we run the algorithm twice, then the number of edge contractions will be two times the number of contractions in a single run, $2(n-2)$. 

The probability of missing a min cut in both runs will be bounded by 

\[
  \Big( 1 - {2 \over n(n-1)}  \Big)^2
\]

Thus, the probability of finding a min cut with two runs will be bounded by

\[ 
1 - \Big( 1 - {2 \over n(n-1)}  \Big)^2
\]

%%%%%%%%%%%%%%%%%%%%%%%%%%%%%%%%%%%%%%%%%%%%%%%%%%%%%

\subsection*{(b)} Solutions: When we contract the graph from $n$ vertices down to $k$ vertices, the number of contraction is $n-k$. For the second step, the number of contractions is $l(k-2)$. So the total number of contraction is $(n-k) + l(k-2)$. 

Assume that the number of edges in the min-cut set $C$ is $m$. Let $E_i$ be the event that the edge contracton contracted in iteration $i$ is not in $C$, adn let $F_i = \cap_{j+1}^i E_j$. In the first contraction, the probability of not choosing an edge in the min-cut set $C$ to contract 
$$
Pr[E_1] = Pr[F_1] \geq 1 - {2m \over nm} = 1 - {2\over n}.
$$

Conditioned on the event that in the first contraction, the edge contracted is not in $C$, we can continue the analysis,

$$
Pr[E_2|F_1] \geq 1 - {m \over m(n-1)/2} = 1 - {2 \over n-1}. 
$$

Similarly,

$$
Pr[E_i|F_{i-1}] \geq 1 - {m \over m(n-i+1)/2} = 1 - {2\over n-i+1}.
$$

Thus, the probability that the edges in the set $C$ have never been contracted until we are left with $k$ vertices is 

\begin{align*}
  Pr[F_{n-k}]   & = Pr[E_{n-k}|F_{n-k-1}] \cdot Pr[E_{n-k-1}|F_{n-k-2}] \cdots Pr[E_2|F_1]\cdot Pr[F_1] \\
                & \geq \prod_{i=1}^{n-k} \Bigg( 1 - {2 \over n-i+1} \Bigg) = \prod_{i=1}^{n-k} \Bigg( {n-i-1 \over n-i+1} \Bigg) \\
                & = {k(k-1) \over n(n-1)}
\end{align*}

Similarly for the second stage, for each copy of the $l$ problems, we can apply the same analysis and bound the probability of finding the min-cut set of size $m$ by 

$$
{2 \over k(k-1)}
$$

Thus, the totally probability of finding the min-cut set after the two steps is bounded by 
$$
{k(k-1) \over n(n-1)} \Bigg(1 - (1-{2 \over k(k-1)})^l\Bigg)
$$

%%%%%%%%%%%%%%%%%%%%%%%%%%%%%%%%%%%%%%%%%%%%%%%%%%%%%

\subsection*{(c)} Solutions: The problem can be formulated as the following optimization problem,

\begin{align*}
  \max_{k, l} \text{   }& {k(k-1) \over n(n-1)} \Bigg(1 - (1-{2 \over k(k-1)})^l\Bigg) \\
  \text{s.t.   } & (n-k) + l(k-2) = 2(n-2)
\end{align*}





\end{document} 

















































