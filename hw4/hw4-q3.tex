\documentclass[10pt]{537homework}

% For including image files
\usepackage{graphicx}
\usepackage[ruled,vlined,noline]{algorithm2e}
% set the vertical spacing between paragraphs
\setlength{\parskip}{1.5mm}

% For fancy math
\RequirePackage{amsmath,amsthm,amssymb}
\newtheorem{theorem}{Theorem}
\newtheorem{fact}[theorem]{Fact}
\newtheorem{lemma}[theorem]{Lemma}
\newtheorem{claim}[theorem]{Claim}
%\usepackage{amsmath}
\usepackage{titlesec}
\titleformat{\subsection}[runin]{}{}{}{}[]

\newcommand{\ord}[2][th]{\ensuremath{{#2}^{\mathrm{#1}}}}
% shorthand for \mathcal{O}
\newcommand{\Ocal}{\ensuremath{\mathcal{O}}}


% homework number
\hwnumber{4}
% problem number
\problemnumber{3}
% your name
\author{Peilun Dai}
% Collaborators. If you didn't collaborate, write "\collaborators{none}".
% If you did, for each collaborator, write "worked together", "I helped him/her" or "He/she helped me".
 \collaborators{None}


\begin{document}
\section*{3. (Consecutive ones) }

%%%%%%%%%%%%%%%%%%%%%%%%%%%%%%%%%%%

\subsection*{(a)} 

Solution: Let R.V. $R_2$ denote the number of rolls until getting a pair of ones, and $X_i$ be the random variable that denote the result at the $i^{th}$ roll. Then by law of total probability and linearity of expectation, we have
\begin{align}
  \mathbf{E}[R_2]     = &(1-Pr[X_1])(\mathbf{E}[R_2]+1)  \nonumber \\
                        & + (Pr[X_1=1](1-Pr[X_2=1]))(\mathbf{E}[R_2]+2) \nonumber \\
                        & + (Pr[X_1=1]Pr[X_2=1])2 \nonumber \\
                      = & {k-1 \over k}(\mathbf{E}[R_2]) 
                          + {1 \over k}{k-1 \over k}(\mathbf{E}[R_2]+2)
                          + {1 \over k^2}2 
\end{align}
If we solve (1), we can obtain $\mathbf{E}[R_2] = k^2 + k$.

%%%%%%%%%%%%%%%%%%%%%%%%%%%%%%%%%%%

\subsection*{(b)} 

Solution: Similar to the last question, let R.V. $R_3$ denote the number of rolls until getting a triple of consecutive ones. 
\begin{align}
  \mathbf{E}[R_3]   =   & (1-Pr[X_1=1])(\mathbf{E}[R_3]+1)  \nonumber \\
                        & + Pr[X_1=1](1-Pr[X_2=1])(\mathbf{E}(R_3) + 2) \nonumber \\
                        & + Pr[X_1=1]Pr[X_2=1](1-Pr[X_3=1])(\mathbf{E}(R_3) + 3) \nonumber \\
                        & + Pr[X_1=1]Pr[X_2=1]Pr[X_3=1]3
\end{align}
We solve (2) for $\mathbf{E}[R_3] = k^3 + k^2 + k - 3$. 

\end{document} 

















































